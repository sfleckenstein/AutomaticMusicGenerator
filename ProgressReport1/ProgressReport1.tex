\documentclass{article}

\usepackage{fullpage}
\usepackage{graphicx}
\usepackage{amsfonts, amsmath}
\usepackage{url}
\usepackage{hyperref}

\pagenumbering{gobble}

\hypersetup{
	colorlinks=true,
	linkcolor=black
}

\begin{document}

\clearpage
\vspace*{\stretch{2}}
\begin{center}
\begin{minipage}{.6\textwidth}

\title{Automatic Music Generator \\ \vspace{2 pt} \Large{Progress Report 1}}
\author{Sam Fleckenstein (sef44) \\ Ross Nanopoulos (rdn21)}
\maketitle

\end{minipage}
\end{center}
\vspace{\stretch{3}}
\clearpage

\tableofcontents
\newpage

\section{Abstract}
The purpose of this project is to develop an intelligent music composer that will analyze common and popular patterns in music, reason about those patterns, and generate a new piece of music that is significantly different than the analyzed pieces, while still being interesting.

\newpage

\section{Introduction}
What is the process by which humans make music? They study the fundamentals: beats, measures, time
and key signatures, tempo, rhythm. They listen to great composers: Bach, Tchaichovsky, Mahler, Debussy,
Chopin. Somehow this knowledge combined with creativity yields additional, masterful compositions. How
then, does one enable a computer to exhibit this thoughtful creativity?\\
\\
A variety of methods have been proposed for algorithmic composition including hidden Markov models
\cite{SOMETHING GOES HERE}, genetic algorithms \cite{SOMETHING GOES HERE}, and neural networks \cite{SOMETHING GOES HERE}. Additionally, a field known as ”combination theory”
has combined these methods to create more advanced learning and composition algorithms \cite{SOMETHING GOES HERE}. Hidden
Markov Models utilize an element of probability and uncertainty that can lead to much more interesting
compositions \cite{SOMETHING GOES HERE}, which is the primary method of algorithmic composition that the system in this report will
use.\\
\\
The argument can be made that innate creativity plays a large role in being able to compose interest-
ing music. However, a goal of artificial intelligence is to eventually develop systems that can think and
have personalities of their own. Thus, this innate creativity when composing music will develop with more
advanced artificially intelligent systems that can think for themselves and exhibit such behavior.

\section{Application}
\subsection{The Echo Nest Interface}
The backbone of the Echo Nest interface will be The Echo Nest’s large database of music intelligence. The
song parser will utilize The Echo Nest’s API to extract useful song information from the database, which
includes a plethora of aspects including time signature, key, mode, tempo, loudness, duration, end of fade
in, start of fade out, audio fingerprint, timbre, pitch, and loudness. Additionally, The Echo Nest provides
sequenced data as ”musically relevant elements” that include segments, tatums, beats, bars, and sections.
This information will allow the learning agent to discern the myriad dynamics of songs and learn about the
ways in which different songs are composed.

\subsection{Learning Agent}
The job of the learning agent will be to take the raw music data gathered by Echo Nest and discover the
relevant patterns in the music. There are a number of different algorithms that could be used to achieve this
goal, but this project use a hidden Markov model to extract these patterns. This model was chosen because
it can be used to represent processes where not all of the information about a state is known. This is useful
because music is very complex and it is very difficult to determine every variable that goes into determining
what should come next in a song. Another reason that hidden Markov models were chosen for this project
is because they have been successfully applied the automatic generation of music \cite{SOMETHING GOES HERE}. The complexity of a
hidden Markov model is also very easy to expand. This can be done by looking at data that is farther in
the past from the current observation, or by adding in more variables to the states you are considering

\subsection{Composition Agent}
The composition agent will take the information that the learning agent provides and decide which notes,
patterns, rhythms, etc. to incorporate into its own piece of music. It will then be responsible for outputting
this generated music to a .wav file, which can be used later for further analysis.

\section{Methodology}
Talk about the clustering that you are/may do. How you are going to do it, how it will be useful,
etc.\\
\\
The learning agent will use the GHMM library \cite{GHMM} to perform the required machine learning.
 This library was chosen because it offers all of the features needed to perform the required 
learning task. It is also free, another important feature.

\section{Software Design}
\subsection{UI}
\begin{enumerate}
\item The UI will prompt the user for a musical genre
\item The UI will prompt the user for a song tempo
\item The UI will prompt the user for a time signature
\item The UI will prompt the user for a key signature
\item The UI will send these user choices to the Echo Nest Interface
\end{enumerate}

\subsection{The Echo Nest Interface}
\begin{enumerate}
\item The Echo Nest Interface will take as input the user input from the UI
\item The Echo Nest Interface will make a call to The Echo Nest API using the user input
\item The Echo Nest will output JSON objects to the Learning Agent
\end{enumerate}

\subsection{Learning Agent}
\begin{enumerate}
\item The learning agent will use the GHMM library
\item The learning agent will take as input JSON objects from the Echo Nest Interface
\item The learning agent will train a model \cite{GHMM} using the GHMM library and the input from the Echo Nest Interface
\item The learning agent will output a trained model to the composition agent
\end{enumerate}

\subsection{Composition Agent}
\begin{enumerate}
\item The composition agent will take as input a trained model from the learning agent
\item The composition agent will create an overarching chord progression
\item The composition agent will then fill in notes using the trained model
\item The composition agent will write the resulting composition to disk as a .wav file
\end{enumerate}

\subsection{User Feedback}
\begin{enumerate}
\item The user will be prompted to rate a number of short song clips
\item The user ratings will be tabulated and used to update the learned model based on which musical patterns the user rated highest
\item The user will be presented with one longer song based on the updated musical model
\end{enumerate}

\section{Project Management}
asdf

\section{User Interface}
asdf

\section{Testing and Evaluation}
The testing is currently all manual. Each of the developers is responsible for testing all of their
components to make sure that they work properly, but the present focus of the project is the 
implementation of all required features. Once all of the desired features are implemented, a
more thorough testing procedure will be implemented to ensure the music generator works as designed.

\section{Project Progress}
asdf

\section{Discussion and Conclusions}
asdf

\end{document}
